% [2]: Seite 11
\chapter{Deklaration}

Folgender Abschnitt beschreibt die Vorkenntnisse des Kandidaten und dessen Vorbereitung.

\section{Vorkenntnisse}

Der Lernende hat seit dem Praktikumsbeginn (1. August 2024) mit Ruby on Rails gearbeitet. Auch wurden die Tests in dieser Zeit mit RSpec und Capybara geschrieben und die Code-Qualität mit Rubocop überprüft. Ebenfalls seit dem Praktikumsstart wurde Git zusammen mit Github eingesetzt. Code-Reviews gehören ebenfalls zur täglichen Arbeit (sowohl Code Reviews durchführen wie auch entgegennehmen). SemaphoreCI wird ebenfalls seit Praktikumsbeginn eingesetzt.

\section{Vorarbeiten}

Folgende Arbeiten müssen vorab erledigt werden:

\begin{itemize}
  \item[\checkmark] Boilerplate Redmine-Plugin wird lokal so aufgesetzt, dass gegen Vanilla-Redmine 5.1.x, 6.0.x mit Postgres getestet werden kann.
  \item[\checkmark] GitHub-Repository wird erstellt
  \item[\checkmark] Erstellen des OpenAI API Tokens.
  \item[\checkmark] Einlesen in die \href{https://www.redmine.org/boards/4/topics/45309}{Redmine-Plugin-Dokumentation}
  \item[\checkmark] Einlesen in Vector Search und Embeddings
  \item[\checkmark] Lesen des \href{https://github.com/minitest/minitest}{MiniTest-README}
  \item[\checkmark] Einarbeitung in Software zur Erstellung der Dokumentation
\end{itemize}

\section{Neue Lerninhalte}

\lipsum[13]

\section{Arbeiten in den letzten 6 Monaten}

\lipsum[14]
